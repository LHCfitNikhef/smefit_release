\documentclass{article}
\usepackage{float}
\usepackage{amsmath}
\usepackage{amssymb}
\usepackage{booktabs}
\usepackage[a4paper]{geometry}
\usepackage{array}
\usepackage{hyperref}
\usepackage{xcolor}
\usepackage{multirow}
\usepackage{pdflscape}
\allowdisplaybreaks
\newcolumntype{C}[1]{>{\centering\let\newline\\\arraybackslash\hspace{0pt}}m{#1}}
\usepackage{graphicx}
\usepackage{tabularx}
\geometry{verbose, tmargin = 1.5cm, bmargin = 1.5cm, lmargin = 1cm, rmargin = 1cm}
\usepackage{underscore}
\begin{document}
$\chi^2$ table. Blue color text represents a value that is lower than the SM $\chi^2$ \
            by more than one standard deviation of the $\chi^2$ distribution.\
            Similarly, red color text represents values that are higher than the SM $\chi^2$ by more than one standard deviation.\
            In parenthesis is the total SM $\chi^2$ for the dataset included in the fit. \\




\begin{table}[H]
\centering
\begin{tabular}{|l|c|c|}
\hline
& \multicolumn{2}{c|}{ATLAS, NS}\\ \hline
Process  & $N_{\rm data}$ & $\chi^2/N_{\rm data}$\\ \hline
 \hline Total & 0.0                 & nan                     (nan) \\ \hline
\end{tabular}
\caption{$\chi^2$ table for grouped data. In parenthesis is the total SM $\chi^2$ for the dataset included in the fit.\
                    The SM column refers to all the datasets available in the group}
\end{table}
\end{document}
